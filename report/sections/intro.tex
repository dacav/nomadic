\emph{Mesh network} are a special kind of \emph{Ad-Hoc network}:
usually they are more structured and tend to present semi-permanent
nodes. However the structure of the network itself is not fixed. Nodes
may join and leave and links between nodes may change during
time. This peculiar characteristic requires a distributed routing
strategy capable of handling such a scenario.

There are two main family of protocols developed to handle routing in
\emph{mesh networks}: \emph{pro-active} and \emph{reactive}.
The former maintain fresh lists of destinations and their routes by
periodically distributing routing tables throughout the network, the
latter finds a route on demand by flooding the network with \emph{route
request packets}.

Thinking about the \emph{pro-active} family a couple of questions
comes to our minds: how the overhead of the routing protocols
affect the network performances and how long is the convergence time
on the event of a sudden link removal?

In this report we will analyze these aspects by showing a series of
results we obtain by running a series of experiments on the
\emph{mesh} routing protocols: \batman\cite{bib:BATMAN} and \olsr\cite{bib:OLSR}.

\subsection{About \batman}

    The \batman\ protocol is based on a \emph{pro-active} approach exploiting a
    flooding technique. Each host in a \batman\ mesh network produces, on a
    regular basis, an \emph{Originator Message} which is sent in
    broadcast and gets received by all directly reachable neighbors. Every
    \emph{Originator Message} contains, among other things, a
    \emph{Sequence Number} and a \emph{TTL} and it is used to
    inform the network of the presence of the node.

    Each node keeps, for any destination host, a sliding window of
    recent sequence numbers. The window gets moved as new sequence numbers
    arrive, and this makes oldest sequence numbers obsolete. The preferred
    route for a specific destination is given by the host having the
    maximum number of entries in the sliding window for the specific
    destination. That is the node which has forwarded, in the current
    window,  the most number of \emph{Originator Messages} generated
    by the destination.

    Any node is in charge of re-broadcasting an \emph{Originator Message} if:
    \begin{itemize}
    \item   The Sequence Number hasn't been seen yet;
    \item   The message's TTL hasn't been expired.
    \end{itemize}

    The suggested period for the flooding operation is of 1 second, thus
    in an environment configured with default values this is the actual
    frequency of the flooding operation.

\subsection{About \olsr}

    The protocol is far more complex than \batman and it's best suited for
    extended and complex networks. Such complexity boils down to
    each host periodically sending a portion of the local \emph{link-state}
    table. The details are beyond the purposes of this report, since
    they're not affecting that much the actual network interaction. What
    really concerns us is the software timing:
    \begin{itemize}
    \item   A probing for the direct neighbors is achieved by means of a
            periodically broadcasted \emph{Hello} message. The
            \Const{HELLO\_INTERVAL} constant, determining the broadcasting
            period, is of 2 seconds;
    \item   Some of the direct neighbors are used as relay nodes, and are
            targeted with messages of type \emph{Topology Control}.
            The \Const{TC\_INTERVAL} constant schedules a direct
            communication every 5 seconds.
    \end{itemize}

\subsection{Experiments with meshed networks}
    As briefly stated above we've executed two kind of tests: the
    first one aims to measure the worsening of the network
    performance in presence of routing protocols w.r.t. a static
    routing configuration while the second measure how
    responsive are the protocols to changes in the network topology.

    The needed data has been retrieved by using some tools:
    \begin{itemize}
    \item   Data about latency has been obtained by means of the
            \emph{ping} tool;
    \item   Throughput has been measured with
            \netperf\cite{bib:NetPerf};
    \item   Statistics about protocol responsiveness for \batman\ and
            \olsr\ have been extrapolated by parsing the output of the
            two softwares respectively.
     \item Complementary data has been extrapolated by the
       \emph{Whireshark} logs
    \end{itemize}

\subsection{The testbed}
   The testbed has been composed by four laptops equipped with
    \emph{Ubuntu 10.10 GNU/Linux} operating systems. We used the following
    softwares:
    \begin{description}
    \item[\netperf] Version \emph{2.4.4-5ubuntu2};
    \item[\batman] Daemon version \emph{0.3.2-5} (userspace version);
    \item[\olsr] Daemon version \emph{0.5.6-r7-1};
    \item[iptables] Version \emph{1.4.4-2ubuntu3};
    \item[wireshark] Version \emph{1.2.11-4build0.10.10.10}.
    \end{description}

    Depending on the tests, the network topologies have been arranged
    specifically, so far each section of this report will be provided with
    a paragraph describing it.
