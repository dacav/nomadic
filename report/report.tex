\documentclass{article}

\usepackage{mathpazo}
\usepackage{listings}
\usepackage{fancyvrb}

% -- PACKAGES -----------------------------------------------------------

\usepackage{times}
\usepackage{graphicx}
\usepackage{verbatim}
\usepackage{amsmath, amssymb}
\usepackage{booktabs}

% TODO have a look:
%\usepackage{subfigure}

\usepackage{color}
\usepackage{url}
\usepackage{a4wide, booktabs}
\usepackage[a4paper,top=2.5cm,bottom=3.0cm,left=2cm,right=2cm,%
            bindingoffset=5mm]{geometry}
\usepackage[bookmarks,
            colorlinks=true,
            linkcolor=blue,
            pdfauthor={Andrea Zito, Giovanni Simoni},
            pdftitle={Report 1: Interference between Bluetooth and 802.11
                      Wireless}]{hyperref}
\usepackage{multicol}
%\newenvironment{multicols}[1]{}{}

% -- COMMANDS -----------------------------------------------------------

\newcommand{\batman}{{\textsf BATMAN}}
\newcommand{\olsr}{{\textsf OLSR}}
\newcommand{\netperf}{{\textsf NetPerf}}

\newcommand{\station}[1]{{\textsf Station #1}}
\newcommand{\primitive}[1]{{\footnotesize \tt #1()}}
\newcommand{\library}[1]{{\textsf #1}}

\newcommand{\Const}[1]{{\tt {#1}}}

% Units of measurement:
\newcommand{\Sec}{s}
\newcommand{\uSec}{\mu\Sec}
\newcommand{\MHz}{MHz}
\newcommand{\GHz}{GHz}
\newcommand{\Slot}{\Variable{slot}}
\newcommand{\Frame}{\Variable{frame}}
\newcommand{\FrameSlot}{\Frame/\Slot}

\newcommand{\Bit}{\mbox{\footnotesize bit}}
\newcommand{\Byte}{\mbox{\footnotesize byte}}

\newcommand{\MBit}{M\Bit}
\newcommand{\MByte}{M\Byte}

\newcommand{\MBitSec}{\MBit/\Sec}
\newcommand{\MByteSec}

\newcommand{\KByteSec}{k\Byte/\Sec}
\newcommand{\KBitSec}{k\Bit/\Sec}

% Probability
\newcommand{\RandomDist}{\sim}
\newcommand{\Probability}[1]{\mbox{{\sf P}}\left(#1\right)}
\newcommand{\Average}[1]{\mathbb{E}\left[#1\right]}
\newcommand{\Variance}[1]{\mathbb{V}\left[#1\right]}
\newcommand{\Normal}[2]{\mathcal{N}\left(#1, #2\right)}
\newcommand{\Interval}[2]{\left[#1, #2\right]}

% Other stuff
\newcommand{\HRule}{\rule{\linewidth}{0.1mm}}
\newcommand{\Ceiling}[1]{\left\lfloor#1\right\rfloor}

% Image

% Parmeters:
% (1) file
% (2) width
% (3) caption
% (4) label
\newcommand{\Picture}[4]{
    \end{multicols}
    \begin{figure}[htbp]
    \centering
    \includegraphics[width={#2}]{{#1}}
    \caption{{#3}}
    \label{{#4}}
    \end{figure}
    \begin{multicols}{2}
}



\begin{document}
  \begin{titlepage}
    \centering

     \begin{figure}[ht]
       \centering
       \includegraphics[angle=-90, keepaspectratio=true, width=8cm]
                       {images/logo}
     \end{figure}

     { \large \bfseries Nomadic Comunication\par AA 2009-2010 }\par

     \vspace{1.5cm}
     { \Large \bfseries \textcolor{blue}
        {Report 2: Mesh Networks}
        \par }
     \vspace{1.0cm}
     { \large \bfseries {Group N. ?} \par }
     \vspace{0.3cm}

     {\large \bfseries {Andrea Zito, Giovanni Simoni}}

     \vspace{1.0cm}

\begin{abstract}

    In this report we are going to analyze and compare the behaviour of
    two different implementation of mesh networking routing protocols.

\end{abstract}

\end{titlepage}

\thispagestyle{empty}
\tableofcontents

\clearpage
\setcounter{page}{1}

\section{Introduction} \label{sec:Intro}
\vspace{-3mm}\HRule

    An interesting issue about \emph{meshed networks} is how they
influence the network performances with respect to a network in which
routes are defined.

\subsection{About \batman}

    * Jitter (collisions)
    * Modulo jitter sono sovrapposti -> worst case
        piu` realisticamente i secondi non sono sovrapposti
    * Best suited for little networks

\subsection{About \olsr}

    \paragraph{Default Constants in \olsr}

        Functioning:
        \begin{itemize}
        \item   A probing for the direct neighbors is achieved by means of
                a periodically sent \emph{Hello} message, which is
                broadcasted. Some of the direct neighbors are used as
                relay nodes;
        \item   The actual \emph{Topology Control};
        \end{itemize}

        Best suited for extended networks

        Reported by section 18 of the rfc:

        \begin{itemize}
        \item   \Const{HELLO\_INTERVAL} of 2 seconds;
        \item   \Const{TC\_INTERVAL} of 5 seconds
        \end{itemize}

\subsection{Experiments with meshed networks}

    The comparison has been achieved by means of two different classes
    of tests: the first one aims to measure the worsening of the network
    performance due to the two protocols, the second one tests how
    responsive are the protocols with respect to changes in the
    network topology.

    The needed data has been retrieved by using some tools:
    \begin{itemize}
    \item   Data about latency has been obtained by means of the
            \emph{ping} tool;
    \item   Throughput has been measured with
            \emph{NetPerf}\cite{bib:NetPerf};
    \item   Statistics about protocol responsiveness for \batman\ and
            \olsr\ have been extrapolated by parsing the output of the
            two softwares respectively.
    \end{itemize}

\subsection{The testbed}

    The testbed has been composed by four laptops equipped with
    (\$distro, \$kernel, \$specifiche) yadda yadda. Basing on the
    different test, the topology has been arranged specifically, so
    far each of the following sections will be provided with a
    paragraph describing it.



\section{Impact on throughput} \label{sec:ImpactThroughput}
\vspace{-3mm}\HRule

    \subsection{Test introduction}

    The throughput is probably the most relevant index of the network
    quality. In the context of \emph{mesh networking} a natural concern
    is how the overhead coming from the underlaying routing protocol
    affects it.

    As we previously mentioned [did we?], both the protocols we worked
    with are of the \emph{proactive} family: the hosts keep sending,
    with a certain periodicity, messages about the topology: the protocol
    related traffic increases along with the number of hosts composing the
    network.  Our testbed was composed by just a few laptops, so a
    reasonable expectation from this test is a negligible variation of the
    performances.

    The core of the experiment consisted in a simple throughput test with
    \netperf\ on three different network topologies. For both the
    analyzed meshing protocols we performed the test while the protocol's
    software was running. We also run an instance of the test with
    statically compiled routes in order to obtain an \emph{ideal overhead
    situation} to be used as term of comparison.

\subsection{Test with direct link}

    \subsubsection{Topology}

        The \emph{direct link} topology is the simplest. Only two laptops
        have been enabled (see Picture~\ref{pic:LayoutDirect}):
        \begin{itemize}
        \item   The laptop sending data (address 10.0.0.65);
        \item   The laptop receiving data (address 10.0.0.67).
        \end{itemize}

        In this situation the message exchange is very limited: every
        synchronization step consists in the two hosts simply exchanging a
        few UDP packets (for a more detailed description refer to
        Section\ref{sec:Intro}).

        The network performances measure confirms our hypothesis: as shown
        in Table~\ref{tab:ThrDirect} the performaces variation is so
        minimal that is more likely to be attributed to the different
        condition of the wireless channel among experiments, rather then
        the actual overhead of the protocols.

        \Picture{images/direct}
                {.49\columnwidth}
                {Configuration with single direct link}
                {pic:LayoutDirect}

        \Picture{images/throughput_plot_direct}
                {0.7 \columnwidth}
                {Impact of the meshing protocols on the throughput of the
                 direct link topology. The three boxplots show,
                 respectively, the performance with \emph{static routes},
                 \emph{\batman} andf \emph{\olsr}}
                {pic:ThpDirect}

        \begin{table}[htbp]
            \centering
            \begin{tabular}{rcccccccc}
            \toprule
            Route & Average (Mbit/s) & Variance & Min & First Quartile &
            Median & Third Quartile & Max & Comp. Static \\
            \midrule
            Static      & 16.970 & 0.191 & 15.74 & 16.67 & 16.97 & 17.26
                        & 18.08  & - \\
            \batman\    & 16.926 & 0.251 & 15.41 & 16.58 & 16.94 & 17.3
                        & 18.4   & 0.997 \\
            \olsr\      & 16.990 & 0.313 & 14.98 & 16.65 & 17.02 & 17.35
                        & 18.86  & 1.001 \\
            \bottomrule
            \end{tabular}
            \label{tab:ThrDirect}
        \end{table}

\subsection{Test with 1 hop}

    \subsubsection{Topology}

        \Picture{images/1hop}
                {.90\columnwidth}
                {Configuration with 1 hop}
                {pic:Layout1Hop}

        \Picture{images/throughput_plot_1hop}
                {0.7 \columnwidth}
                {Impact of the meshing protocols on the throughput of the
                 1-hop topology. The three boxplots show, respectively, the
                 performance with \emph{static routes}, \emph{\batman} and
                 \emph{\olsr}}
                {pic:Thp1Hop}

        \begin{table}[htbp]
            \centering
            \begin{tabular}{rcccccccc}
            \toprule
            Route & Average (Mbit/s) & Variance & Min & First Quartile &
            Median & Third Quartile & Max & Comp. Static \\
            \midrule
            Static      & 8.496 & 1.118 & 2.635 & 8.09 & 8.62 & 9.08
                        & 11.12 & - \\
            \batman\    & 7.973 & 2.197 & 1.455 & 7.425 & 8.27 & 8.855
                        & 11.71 & 0.938 \\
            \olsr\      & 6.924 & 7.68 & 0.096 & 6.91 & 8.01 & 8.62
                        & 10.21 & 0.815 \\
            \bottomrule
            \end{tabular}
            \label{tab:ThrDirect}
        \end{table}

\subsection{Test with 2 hop}

    \subsubsection{Topology}

        \Picture{images/2hop}
                {.90\columnwidth}
                {Configuration with 2 hop}
                {pic:Layout2Hop}

        \Picture{images/throughput_plot_2hop}
                {0.7 \columnwidth}
                {Impact of the meshing protocols on the throughput of the
                 2-hops topology. The three boxplots show, respectively,
                 the performance with \emph{static routes}, \emph{\batman}
                 and \emph{\olsr}}
                {pic:Thp2Hops}

        \begin{table}[htbp]
            \centering
            \begin{tabular}{rcccccccc}
            \toprule
            Route & Average (Mbit/s) & Variance & Min & First Quartile &
            Median & Third Quartile & Max & Comp. Static \\
            \midrule
            Static      & 5.344 & 2.183 & 1.332 & 4.82 & 5.84 & 6.27
                        & 8.22 & - \\
            \batman\    & 5.381 & 2.03 & 0.753 & 4.92 & 5.84 & 6.29
                        & 8.81 & 1.007 \\
            \olsr\      & 5.193 & 3.228 & 0.12 & 5.015 & 5.82 & 6.33
                        & 7.77 & 0.972 \\
            \bottomrule
            \end{tabular}
            \label{tab:ThrDirect}
        \end{table}



\clearpage
\section{Impact on latency} \label{sec:ImpactLatency}
\vspace{-3mm}\HRule

    \clearpage
\subsection{Test introduction}
    Beyond throughput, another measure of interest when analyzing the
    performance of a network is the latency. We've seen in the
    previous section that the presence of the \emph{mesh} routing
    protocol do not seem to affect the overall throughput in the small
    environment at our disposal. An obvious question which could be
    raised is if the same is true also when taking into account latency.

    To answer this question we performed similar test to the one
    described above. For each topology and routing strategy we
    performed a latency test by \emph{pinging} the sink node from the
    destination. Each measurement lasted 600 seconds.

\subsection{Results}
The results we obtained are summarized in
Picture~\ref{pic:Latency}. As it's clear from the image, we have a
situation analogous to the one of throughput. The difference between
the three routing strategy is negligible.

Again an interesting note can be made on the behavior of \olsr. Even
if in the \emph{direct} topology it obtained the best result (though
with really spread min-max bounds) in the remaining experiments it
achieves worse performance. \batman\ instead is more constant and
always scores results comparable with the \emph{static} setting.

\Picture{images/latency_plot}
        {0.7 \columnwidth}
        {Impact of the meshing protocols on the transmission latency. The
         three triples of boxplots show the latency performances of,
         respectively, \emph{direct link}, \emph{1-hop} and \emph{2-hops}
         topology. Each triple shows the performance with \emph{static
         routes} (first boxplot), \emph{\batman} (second boxplot) and
         \emph{\olsr} (third boxplot).}
        {pic:Latency}


\clearpage
\section{Convergence time} \label{sec:Convergence}
\vspace{-3mm}\HRule

    \Picture{images/convergence}
        {.90\columnwidth}
        {Configuration with single direct link}
        {pic:LayoutConvergence}




\section{Conclusions} \label{sec:Conclusions}
\vspace{-3mm}\HRule

    Our experiments gave us a positive opinion about both the analyzed
protocols. They are obviously able to make the mesh network work in a
sound way and also without a perceptible impact on the transmission
quality: the worsening of both latency and throughput are
negligible. That said however we weren't able to simulate a complex scenario
which would have emphatized their weakness: the number of nodes we
employed was just too small.

As a final remark, our impression was that \batman\ seems to be more
stable and overall shows more constant performances. This is probably
due to its simpler structure w.r.t. \olsr. However this consideration
may be influenced by the fact that our testbed is quite simple too. A
more complex configuration might work in favor of \olsr\ which is
designed to work on such topologies.

\bibliographystyle{plain}

\begin{thebibliography}{99}

    \bibitem{bib:BATMAN} The \batman\ protocol:
    \url{http://www.open-mesh.org/}

    \bibitem{bib:OLSR} The \olsr\ protocol:
    \url{http://www.olsr.org/}

    \bibitem{bib:NetPerf} The \netperf\ tool (network performance
    measure)

%    \bibitem{bib:WirelessSpec} IEEE Std 802.11\texttrademark 2007, {\it
%        Part 11: Wireless LAN Medium Access Control (MAC) and Physical
%        Layer (PHY) Specifications}
%
%    % TODO: add info about artelatex and other used documents.
%
%    \bibitem{} Eric W. Weisstein,  ``Confidence Interval,'' From
%    MathWorld--A Wolfram Web Resource,
%    \url{http://mathworld.wolfram.com/ConfidenceInterval.html}

\end{thebibliography}

\end{document}
