An interesting issue about \emph{meshed networks} is how they influence
the network performances with respect to configurations in which routes are
defined statically. The comparison we worked on takes in account
\batman\cite{bib:BATMAN} and \olsr\cite{bib:OLSR}, both implementing mesh
routing protocols.

\subsection{About \batman}

    The \batman\ protocol is based on a proactive approach exploiting a
    flooding technique. Each host in a \batman\ mesh network produces, on a
    regular basis, an \emph{Originator Message} which is sent in
    broadcast and gets received by all directly reachable neighbors. Every
    \emph{Originator Message} contains, among other things, a
    \emph{Sequence Number} and a \emph{TTL}.

    Each node keeps, for any destination host, a sliding window of
    recent sequence numbers. The window gets moved as new sequence numbers
    arrive, and this makes oldest sequence numbers obsolete. The preferred
    route is given by the host having the maximum number of entries in the
    sliding window.  So far, each packet contains only a few data, and
    doesn't give any kind of information on how the routing table is
    structured in the sender Originator.

    The node is in charge of re-broadcasting a message from another node
    if:
    \begin{itemize}
    \item   The Sequence Number hasn't been seen yet;
    \item   The message's TTL hasn't been expired.
    \end{itemize}

    The suggested period for the flooding operation is of 1 second, thus
    in an environment configured with default values this is the actual
    frequency of the flooding operation.

\subsection{About \olsr}

    The protocol is far more complex than \batman, as it's best suited for
    extended and complex networks. Such complexity boils down to
    each host periodically sending of a part of the \emph{link-state}
    table. The details are beyond the purposes of this report, since
    they're not affecting that much the actual network interaction. What
    really concerns us is the software timing:
    \begin{itemize}
    \item   A probing for the direct neighbors is achieved by means of a
            periodically broadcasted \emph{Hello} message. The
            \Const{HELLO\_INTERVAL} constant, determining the broadcasting
            period, is of 2 seconds;
    \item   Some of the direct neighbors are used as relay nodes, and are
            targeted with messages of type \emph{Topology Control}.
            The \Const{TC\_INTERVAL} constant schedules a direct
            communication every 5 seconds.
    \end{itemize}

\subsection{Experiments with meshed networks}

    The comparison has been achieved by means of two different classes
    of tests: the first one aims to measure the worsening of the network
    performance due to the two protocols, the second one tests how
    responsive are the protocols with respect to changes in the
    network topology.

    The needed data has been retrieved by using some tools:
    \begin{itemize}
    \item   Data about latency has been obtained by means of the
            \emph{ping} tool;
    \item   Throughput has been measured with
            \netperf\cite{bib:NetPerf};
    \item   Statistics about protocol responsiveness for \batman\ and
            \olsr\ have been extrapolated by parsing the output of the
            two softwares respectively.
    \end{itemize}

\subsection{The testbed}

    The testbed has been composed by four laptops equipped with
    \emph{Ubuntu 10.10 GNU/Linux} operating systems. We used the following
    softwares:
    \begin{description}
    \item[\netperf] Version \emph{2.4.4-5ubuntu2};
    \item[\batman] Daemon version \emph{0.3.2-5} (userspace version);
    \item[\olsr] Daemon version \emph{0.5.6-r7-1};
    \item[iptables] Version \emph{1.4.4-2ubuntu3};
    \item[wireshark] Version \emph{1.2.11-4build0.10.10.10}.
    \end{description}

    Depending on the tests, the network topologies have been arranged
    specifically, so far each section of this report will be provided with
    a paragraph describing it.

