An interesting issue about \emph{meshed networks} is how they
influence the network performances with respect to a network in which
routes are defined.

\subsection{About \batman}

    The \batman\ protocol is based on a proactive approach exploiting a
    flooding technique. Each host in a \batman\ mesh network produces, on a
    regular basis, an \emph{Originator Message} which is sent in
    broadcast, thus is received by all directly reachable neighbors. Every
    \emph{Originator Message} contains, among other things, a sequence
    number and a TTL. Each node is in charge of retransmitting a message
    from another node if:
    \begin{itemize}
    \item   The sequence number hasn't been seen yet;
    \item   The message's TTL hasn't been expired.
    \end{itemize}

    The suggested period for the flooding operation is of 1 second, thus
    in an environment configured with default values this is the actual
    frequency of the flooding operation.

%    In order to reduce the probability of collision with other
%    Originators, the software exploits a random jitter in the message
%    sending.

%    * Jitter (collisions)
%    * Modulo jitter sono sovrapposti -> worst case
%        piu` realisticamente i secondi non sono sovrapposti
%    * Best suited for little networks

\subsection{About \olsr}

    The protocol is far more complex than \batman, and it's best suited
    for extended and complex networks. All the complexity, however, is
    related to the internal implementation, and it doesn't affect that
    much the actual network interaction. What really concerns the network
    is the software timing:
    \begin{itemize}
    \item   A probing for the direct neighbors is achieved by means of a
            periodically sent \emph{Hello} message, which is broadcasted.
            The \Const{HELLO\_INTERVAL} constant, determining the
            broadcasting period is of 2 seconds;
    \item   Some of the direct neighbors are used as relay nodes, and are
            targeted with the \emph{Topology Control} messages.
            The \Const{TC\_INTERVAL} constant, determining the direct
            communication of 5 seconds
    \end{itemize}

\subsection{Experiments with meshed networks}

    The comparison has been achieved by means of two different classes
    of tests: the first one aims to measure the worsening of the network
    performance due to the two protocols, the second one tests how
    responsive are the protocols with respect to changes in the
    network topology.

    The needed data has been retrieved by using some tools:
    \begin{itemize}
    \item   Data about latency has been obtained by means of the
            \emph{ping} tool;
    \item   Throughput has been measured with
            \netperf\cite{bib:NetPerf};
    \item   Statistics about protocol responsiveness for \batman\ and
            \olsr\ have been extrapolated by parsing the output of the
            two softwares respectively.
    \end{itemize}

\subsection{The testbed}

    The testbed has been composed by four laptops equipped with
    (\$distro, \$kernel, \$specifiche) yadda yadda. Basing on the
    different test, the topology has been arranged specifically, so
    far each of the following sections will be provided with a
    paragraph describing it.
