\subsection{Test introduction}
    Beyond throughput, another measure of interest when analyzing the
    performance of a network is the latency. We've seen in the
    previous section that the presence of the \emph{mesh} routing
    protocol doesn't seem to affect the overall throughput in the small
    environment we worked with. An obvious question which could be
    raised is if the same is true also when taking into account latency.

    To answer this question we performed similar test to the one
    described above. For each topology and routing strategy we
    performed a latency test by \emph{pinging} for 600 seconds the sink
    node from the source.

\subsection{Results and considerations}

    The results we obtained are summarized in
    Picture~\ref{pic:Latency}. As it's visually clear from the image, we
    have a situation analogous to the one of throughput. The difference
    between the three routing strategy is negligible, while the latency
    grows pointedly as we force the communication through more hops.
    The precise values of the data represented graphically are available,
    for the \emph{direct}, \emph{1-hop} and \emph{2-hops} topologies
    respectively, in Table~\ref{tab:LatDirect}, \ref{tab:Lat1Hop} and
    \ref{tab:Lat2Hop}.

    Again an interesting note can be made on the behavior of \olsr. Even
    if in the \emph{direct} topology it obtained the best result (though
    with really spread min-max bounds) in the remaining experiments it
    achieves worse performance. \batman\ instead is more constant and
    always scores results comparable with the \emph{static} setting.

\Picture{images/latency_plot}
        {0.7 \columnwidth}
        {Impact of the meshing protocols on the transmission latency. The
         three triples of boxplots show the latency performances of,
         respectively, \emph{direct link}, \emph{1-hop} and \emph{2-hops}
         topology. Each triple shows the performance with \emph{static
         routes} (first boxplot), \emph{\batman} (second boxplot) and
         \emph{\olsr} (third boxplot).}
        {pic:Latency}

\begin{table}[htbp]
    \centering
    \begin{tabular}{rccccccc}
    \toprule
    Protocol & Average & Variance & Min & 1st Quartile &
    Median & 3rd Quartile & Max \\
    & \footnotesize{\MilliSec} & & \footnotesize{\MilliSec} & \footnotesize{\MilliSec} &
    \footnotesize{\MilliSec} & \footnotesize{\MilliSec} & \footnotesize{\MilliSec} \\

    \midrule
    static & 0.692 & 0.001 & 0.654 & 0.671 & 0.686 & 0.706 & 0.764 \\
    batman & 0.68 & 0.0 & 0.647 & 0.665 & 0.679 & 0.693 & 0.721 \\
    olsr & 0.63 & 0.009 & 0.487 & 0.571 & 0.615 & 0.661 & 1.2 \\
    \bottomrule
    \end{tabular}
    \caption{Latency results for direct topology.}
    \label{tab:LatDirect}
\end{table}

\begin{table}[htbp]
    \centering
    \begin{tabular}{rccccccc}
    \toprule
    Protocol & Average & Variance & Min & 1st Quartile &
    Median & 3rd Quartile & Max \\
    & \footnotesize{\MilliSec} & & \footnotesize{\MilliSec} & \footnotesize{\MilliSec} &
    \footnotesize{\MilliSec} & \footnotesize{\MilliSec} & \footnotesize{\MilliSec} \\

    \midrule
    static & 1.178 & 0.048 & 0.956 & 1.06 & 1.11 & 1.18 & 2.32 \\
    batman & 1.163 & 0.043 & 0.946 & 1.04 & 1.1 & 1.19 & 2.12 \\
    olsr & 1.206 & 0.047 & 0.952 & 1.09 & 1.15 & 1.21 & 2.38 \\

    \bottomrule
    \end{tabular}
    \caption{Latency results for 1 hop topology.}
    \label{tab:Lat1Hop}
\end{table}

\begin{table}[htbp]
    \centering
    \begin{tabular}{rccccccc}
    \toprule
    Protocol & Average & Variance & Min & 1st Quartile &
    Median & 3rd Quartile & Max \\
    & \footnotesize{\MilliSec} & & \footnotesize{\MilliSec} & \footnotesize{\MilliSec} &
    \footnotesize{\MilliSec} & \footnotesize{\MilliSec} & \footnotesize{\MilliSec} \\

    \midrule
    static & 1.676 & 0.062 & 1.41 & 1.54 & 1.6 & 1.67 & 2.92 \\
    batman & 1.639 & 0.079 & 1.36 & 1.48 & 1.55 & 1.66 & 3.2 \\
    olsr & 1.745 & 0.08 & 1.4 & 1.58 & 1.65 & 1.815 & 3.09 \\
    \bottomrule
    \end{tabular}
    \caption{Latency results for 2 hops topology.}
    \label{tab:Lat2Hop}
\end{table}


