\subsection{Test introduction}

    The throughput is probably the most relevant index of the network
    quality. In the context of \emph{mesh networking} a natural concern
    is how the overhead coming from the underlaying routing protocol
    affects it.

    As we previously mentioned [did we?], both the protocols we worked
    with are of the \emph{proactive} family: the hosts keep sending,
    with a certain periodicity, messages about the topology: the protocol
    related traffic increases along with the number of hosts composing the
    network.  Our testbed was composed by just a few laptops, so a
    reasonable expectation from this test is a negligible variation of the
    performances.

    The core of the experiment consisted in a simple throughput test with
    \netperf\ on three different network topologies. For each of the
    analyzed meshing protocols we achieved the test while the meshing
    protocol was running. We also run an instance of the test with
    statically compiled routes in order to obtain an \emph{ideal overhead
    situation} to be used as term of comparison.

\subsection{Test with direct link}

    \subsubsection{Topology}

        The \emph{direct link} topology is the simplest. Only two laptops
        have been enabled (see Picture~\ref{pic:LayoutDirect}):
        \begin{itemize}
        \item   The laptop sending data (address 10.0.0.65);
        \item   The laptop receiving data (address 10.0.0.67).
    \end{itemize}

        \Picture{images/direct}
                {.49\columnwidth}
                {Configuration with single direct link}
                {pic:LayoutDirect}

        \Picture{images/throughput_plot_direct}
                {0.7 \columnwidth}
                {Impact of the meshing protocols on the throughput of the
                 direct link topology. The three boxplots show,
                 respectively, the performance with \emph{static routes},
                 \emph{\batman} andf \emph{\olsr}}
                {pic:ThpDirect}

        \begin{tabular}{rc}
        \toprule
        Route definition & Throughput (Mbit/sec) \\
        \midrule
        Static route    & 16.932 \\
        \batman\        & 16.883 \\
        \olsr\          & 16.957 \\
        \bottomrule
        \end{tabular}

\subsection{Test with 1 hop}

    \subsubsection{Topology}

        \Picture{images/1hop}
                {.90\columnwidth}
                {Configuration with 1 hop}
                {pic:Layout1Hop}

        \Picture{images/throughput_plot_1hop}
                {0.7 \columnwidth}
                {Impact of the meshing protocols on the throughput of the
                 1-hop topology. The three boxplots show, respectively, the
                 performance with \emph{static routes}, \emph{\batman} and
                 \emph{\olsr}}
                {pic:Thp1Hop}

\subsection{Test with 2 hop}

    \subsubsection{Topology}

        \Picture{images/2hop}
                {.90\columnwidth}
                {Configuration with 2 hop}
                {pic:Layout2Hop}

        \Picture{images/throughput_plot_2hop}
                {0.7 \columnwidth}
                {Impact of the meshing protocols on the throughput of the
                 2-hops topology. The three boxplots show, respectively,
                 the performance with \emph{static routes}, \emph{\batman}
                 and \emph{\olsr}}
                {pic:Thp2Hops}
