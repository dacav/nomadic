\subsection{Test introduction}

    The throughput is probably the most relevant index of the network
    quality. In the context of \emph{mesh networking} a natural concern
    is how the overhead coming from the underlaying routing protocol
    affects it.

    As we previously mentioned [did we?], both the protocols we worked
    with are of the \emph{proactive} family: the hosts keep sending,
    with a certain periodicity, messages about the topology: the protocol
    related traffic increases along with the number of hosts composing the
    network.  Our testbed was composed by just a few laptops, so a
    reasonable expectation from this test is a negligible variation of the
    performances.

    The core of the experiment consisted in a simple throughput test with
    \netperf\ on three different network topologies. For both the
    analyzed meshing protocols we performed the test while the protocol's
    software was running. We also run an instance of the test with
    statically compiled routes in order to obtain an \emph{ideal overhead
    situation} to be used as term of comparison.

\subsection{Test with direct link}

\subsubsection{Topology}
        The \emph{direct link} topology is the simplest. Only two laptops
        have been enabled (see Picture~\ref{pic:LayoutDirect}):
        \begin{itemize}
        \item   The laptop sending data (address 10.0.0.65);
        \item   The laptop receiving data (address 10.0.0.67).
        \end{itemize}
        
        \Picture{images/direct}
                {.49\columnwidth}
                {Configuration with single direct link}
                {pic:LayoutDirect}

\subsubsection{Results}
        In this situation the message exchange is very limited: every
        synchronization step consists in the two hosts simply exchanging a few
        UDP packets (for a more detailed description refer to
        Section~\ref{sec:Intro}). The network performances measure confirms our
        hypothesis: as shown in Table~\ref{tab:ThrDirect} the performaces
        variation is so minimal that is more likely to be attributed to the
        different condition of the wireless channel among experiments, rather
        then the actual overhead of the protocols.

        \Picture{images/throughput_plot_direct}
                {0.7 \columnwidth}
                {Impact of the meshing protocols on the throughput of the
                 direct link topology. The three boxplots show,
                 respectively, the performance with \emph{static routes},
                 \batman\ and \olsr.}
                {pic:ThpDirect}

        \begin{table}[htbp]
            \centering
            \begin{tabular}{rcccccccc}
            \toprule
            Protocol & Average & Variance & Min & 1st Quartile &
            Median & 3rd Quartile & Max & Comp. wr.t.\\
            & \footnotesize{\MBitsSec} & & \footnotesize{\MBitsSec} & \footnotesize{\MBitsSec} &
            \footnotesize{\MBitsSec} & \footnotesize{\MBitsSec} & \footnotesize{\MBitsSec} & Static\\

            \midrule
            Static      & 16.970 & 0.191 & 15.74 & 16.67 & 16.97 & 17.26
                        & 18.08  & - \\
            \batman\    & 16.926 & 0.251 & 15.41 & 16.58 & 16.94 & 17.3
                        & 18.4   & 0.997 \\
            \olsr\      & 16.990 & 0.313 & 14.98 & 16.65 & 17.02 & 17.35
                        & 18.86  & 1.001 \\
            \bottomrule
            \end{tabular}
            \caption{Throughput result for direct topology.}
            \label{tab:ThrDirect}
        \end{table}

\subsection{Test with 1 hop}
\subsubsection{Topology}
        The \emph{1 hop} topology is composed by a 3 computer disposed
        in a chain (see Picture~\ref{pic:Layout1Hop}):
        \begin{itemize}
        \item   The laptop sending data (address 10.0.0.65);
        \item   The laptop acting as hop (address 10.0.0.66);
        \item   The laptop receiving data (address 10.0.0.67).
        \end{itemize}

        \Picture{images/1hop}
                {.90\columnwidth}
                {Configuration with 1 hop}
                {pic:Layout1Hop}

        To implement such a scheme we have to prevent  the source
        from communicating directly with the sink and vice versa. Given
        that we're operating with a wireless medium it isn't possible
        to physically stop the 2 computer from seeing each
        other. Instead we exploit \emph{iptables} to filter out incoming
        packet based on the MAC address of the sender.
        
        \begin{verbatim}
on source:
iptables -A INPUT -m mac --mac-source $(sinkMAC) -j DROP;

on sink:
iptables -A INPUT -m mac --mac-source $(sourceMAC) -j DROP;
    \end{verbatim}

\subsubsection{Results}
      As before is reasonable to assume the overhead imposed by
      the routing protocols to be negligible, as the number of node is
      really small. Again we expected to measure equivalent average
      throughputs. Instead the results we obtained show otherwise. 
      As we can clearly see in  Picture~\ref{pic:Thp1Hop}, when using
      \batman\ and \olsr\, the average
      throughput decreases significantly while the variance increases notably.

      In our opinion this result does not provide an attendible
      picture of the reality. This consideration is sustained both by
      our initial hypothesis on the behaviour of the routing protocols
      and by the fact that this is the only experiment in which we
      measured such a high variation (even taking into account the
      result with the 2 hops, see \ref{subsec:2hop}, configuration
      which theoretically should show even more the effect of the overhead).

        \Picture{images/throughput_plot_1hop}
                {0.7 \columnwidth}
                {Impact of the meshing protocols on the throughput of the
                 1-hop topology. The three boxplots show, respectively, the
                 performance with \emph{static routes}, \emph{\batman} and
                 \emph{\olsr}}
                {pic:Thp1Hop}

        \begin{table}[htbp]
            \centering
            \begin{tabular}{rcccccccc}
            \toprule
            Route & Average & Variance & Min & 1st Quartile &
            Median & 3rd Quartile & Max & Comp. w.r.t.\\
            & \footnotesize{\MBitsSec} & & \footnotesize{\MBitsSec} & \footnotesize{\MBitsSec} &
            \footnotesize{\MBitsSec} & \footnotesize{\MBitsSec} & \footnotesize{\MBitsSec} & Static\\
            \midrule
            Static      & 8.496 & 1.118 & 2.635 & 8.09 & 8.62 & 9.08
                        & 11.12 & - \\
            \batman\    & 7.973 & 2.197 & 1.455 & 7.425 & 8.27 & 8.855
                        & 11.71 & 0.938 \\
            \olsr\      & 6.924 & 7.68 & 0.096 & 6.91 & 8.01 & 8.62
                        & 10.21 & 0.815 \\
            \bottomrule
            \end{tabular}
            \caption{Throughput result for 1hop topology}
            \label{tab:ThrDirect}
        \end{table}

\subsection{Test with 2 hop}
\label{subsec:2hop}
    \subsubsection{Topology}
        The \emph{2 hops} topology is composed by a 4 computer disposed
        in a chain (see Picture~\ref{pic:Layout2Hop}):
        \begin{itemize}
        \item   The laptop sending data (address 10.0.0.65);
        \item   The laptop acting as 1st hop (address 10.0.0.66);
        \item   The laptop acting as 2st hop (address 10.0.0.68);
        \item   The laptop receiving data (address 10.0.0.67).
        \end{itemize}

        \Picture{images/2hop}
                {.90\columnwidth}
                {Configuration with 2 hop}
                {pic:Layout2Hop}

         The \emph{iptables} rule we used to implement this
         configuration are the following:

        \begin{verbatim}
on source:
iptables -A INPUT -m mac --mac-source $(sinkMAC) -j DROP;
iptables -A INPUT -m mac --mac-source $(hop2MAC) -j DROP;

on hop1:
iptables -A INPUT -m mac --mac-source $(sinkMAC) -j DROP;

on hop2:
iptables -A INPUT -m mac --mac-source $(sourceMAC) -j DROP;

on sink:
iptables -A INPUT -m mac --mac-source $(sourceMAC) -j DROP;
iptables -A INPUT -m mac --mac-source $(hop1MAC) -j DROP;
    \end{verbatim}

\subsubsection{Results}
        This time the results are more compatible with our
        hypothesis. The throughput difference in the 3 variants is
        negligible and most likely caused by noise of the wireless
        channel.

        \Picture{images/throughput_plot_2hop}
                {0.7 \columnwidth}
                {Impact of the meshing protocols on the throughput of the
                 2-hops topology. The three boxplots show, respectively,
                 the performance with \emph{static routes}, \emph{\batman}
                 and \emph{\olsr}}
                {pic:Thp2Hops}

        \begin{table}[htbp]
            \centering
            \begin{tabular}{rcccccccc}
            \toprule
            Route & Average & Variance & Min & 1st Quartile &
            Median & 3rd Quartile & Max & Comp. w.r.t.\\
            & \footnotesize{\MBitsSec} & & \footnotesize{\MBitsSec} & \footnotesize{\MBitsSec} &
            \footnotesize{\MBitsSec} & \footnotesize{\MBitsSec} & \footnotesize{\MBitsSec} & Static\\
            \midrule
            Static      & 5.344 & 2.183 & 1.332 & 4.82 & 5.84 & 6.27
                        & 8.22 & - \\
            \batman\    & 5.381 & 2.03 & 0.753 & 4.92 & 5.84 & 6.29
                        & 8.81 & 1.007 \\
            \olsr\      & 5.193 & 3.228 & 0.12 & 5.015 & 5.82 & 6.33
                        & 7.77 & 0.972 \\
            \bottomrule
            \end{tabular}
            \caption{Throughput result for 2hop topology}
            \label{tab:ThrDirect}
        \end{table}

