\documentclass{article}

\usepackage{mathpazo}

% -- PACKAGES -----------------------------------------------------------

\usepackage{times}
\usepackage{graphicx}
\usepackage{verbatim}
\usepackage{amsmath, amssymb}
\usepackage{booktabs}

% TODO have a look:
%\usepackage{subfigure}

\usepackage{color}
\usepackage{url}
\usepackage{a4wide, booktabs}
\usepackage[a4paper,top=2.5cm,bottom=3.0cm,left=2cm,right=2cm,%
            bindingoffset=5mm]{geometry}
\usepackage[bookmarks,
            colorlinks=true,
            linkcolor=blue,
            pdfauthor={Andrea Zito, Giovanni Simoni},
            pdftitle={Report 1: Interference between Bluetooth and 802.11
                      Wireless}]{hyperref}
\usepackage{multicol}
%\newenvironment{multicols}[1]{}{}

% -- COMMANDS -----------------------------------------------------------

\newcommand{\batman}{{\textsf BATMAN}}
\newcommand{\olsr}{{\textsf OLSR}}
\newcommand{\netperf}{{\textsf NetPerf}}

\newcommand{\station}[1]{{\textsf Station #1}}
\newcommand{\primitive}[1]{{\footnotesize \tt #1()}}
\newcommand{\library}[1]{{\textsf #1}}

\newcommand{\Const}[1]{{\tt {#1}}}

% Units of measurement:
\newcommand{\Sec}{s}
\newcommand{\uSec}{\mu\Sec}
\newcommand{\MHz}{MHz}
\newcommand{\GHz}{GHz}
\newcommand{\Slot}{\Variable{slot}}
\newcommand{\Frame}{\Variable{frame}}
\newcommand{\FrameSlot}{\Frame/\Slot}

\newcommand{\Bit}{\mbox{\footnotesize bit}}
\newcommand{\Byte}{\mbox{\footnotesize byte}}

\newcommand{\MBit}{M\Bit}
\newcommand{\MByte}{M\Byte}

\newcommand{\MBitSec}{\MBit/\Sec}
\newcommand{\MByteSec}

\newcommand{\KByteSec}{k\Byte/\Sec}
\newcommand{\KBitSec}{k\Bit/\Sec}

% Probability
\newcommand{\RandomDist}{\sim}
\newcommand{\Probability}[1]{\mbox{{\sf P}}\left(#1\right)}
\newcommand{\Average}[1]{\mathbb{E}\left[#1\right]}
\newcommand{\Variance}[1]{\mathbb{V}\left[#1\right]}
\newcommand{\Normal}[2]{\mathcal{N}\left(#1, #2\right)}
\newcommand{\Interval}[2]{\left[#1, #2\right]}

% Other stuff
\newcommand{\HRule}{\rule{\linewidth}{0.1mm}}
\newcommand{\Ceiling}[1]{\left\lfloor#1\right\rfloor}

% Image

% Parmeters:
% (1) file
% (2) width
% (3) caption
% (4) label
\newcommand{\Picture}[4]{
    \end{multicols}
    \begin{figure}[htbp]
    \centering
    \includegraphics[width={#2}]{{#1}}
    \caption{{#3}}
    \label{{#4}}
    \end{figure}
    \begin{multicols}{2}
}



\begin{document}
  \begin{titlepage}
    \centering

     \begin{figure}[ht]
       \centering
       \includegraphics[angle=-90, keepaspectratio=true, width=8cm]
                       {images/logo}
     \end{figure}

     { \large \bfseries Nomadic Comunication\par AA 2009-2010 }\par

     \vspace{1.5cm}
     { \Large \bfseries \textcolor{blue}
        {Report 2: Mesh Networks}
        \par }
     \vspace{1.0cm}
     { \large \bfseries {Group N. ?} \par }
     \vspace{0.3cm}

     {\large \bfseries {Andrea Zito, Giovanni Simoni}}

     \vspace{1.0cm}

\begin{abstract}

    This report documents some tests we ran. Our purpose has been the
    comparsion of two different routing protocols in the context of the
    meshed network.

\end{abstract}

\end{titlepage}

\thispagestyle{empty}
\tableofcontents

\clearpage
\setcounter{page}{1}

\section{Introduction} \label{sec:Intro}
\vspace{-3mm}\HRule

    An interesting issue about \emph{meshed networks} is how they
influence the network performances with respect to a network in which
routes are defined.

\subsection{About \batman}

    * Jitter (collisions)
    * Modulo jitter sono sovrapposti -> worst case
        piu` realisticamente i secondi non sono sovrapposti
    * Best suited for little networks

\subsection{About \olsr}

    \paragraph{Default Constants in \olsr}

        Functioning:
        \begin{itemize}
        \item   A probing for the direct neighbors is achieved by means of
                a periodically sent \emph{Hello} message, which is
                broadcasted. Some of the direct neighbors are used as
                relay nodes;
        \item   The actual \emph{Topology Control};
        \end{itemize}

        Best suited for extended networks

        Reported by section 18 of the rfc:

        \begin{itemize}
        \item   \Const{HELLO\_INTERVAL} of 2 seconds;
        \item   \Const{TC\_INTERVAL} of 5 seconds
        \end{itemize}

\subsection{Experiments with meshed networks}

    The comparison has been achieved by means of two different classes
    of tests: the first one aims to measure the worsening of the network
    performance due to the two protocols, the second one tests how
    responsive are the protocols with respect to changes in the
    network topology.

    The needed data has been retrieved by using some tools:
    \begin{itemize}
    \item   Data about latency has been obtained by means of the
            \emph{ping} tool;
    \item   Throughput has been measured with
            \emph{NetPerf}\cite{bib:NetPerf};
    \item   Statistics about protocol responsiveness for \batman\ and
            \olsr\ have been extrapolated by parsing the output of the
            two softwares respectively.
    \end{itemize}

\subsection{The testbed}

    The testbed has been composed by four laptops equipped with
    (\$distro, \$kernel, \$specifiche) yadda yadda. Basing on the
    different test, the topology has been arranged specifically, so
    far each of the following sections will be provided with a
    paragraph describing it.



\section{Impact of a meshing network protocol} \label{sec:Impact}
\vspace{-3mm}\HRule

    \Picture{images/direct}
        {.49\columnwidth}
        {Configuration with single direct link}
        {pic:LayoutDirect}

\Picture{images/1hop}
        {.90\columnwidth}
        {Configuration with 1 hop}
        {pic:Layout1Hop}

\Picture{images/2hop}
        {.90\columnwidth}
        {Configuration with 2 hop}
        {pic:Layout2Hop}

\begin{multicols}{2}

\end{multicols}




\section{Adding a node} \label{sec:AddNode}
\vspace{-3mm}\HRule

%    \input{sections/testbed}

\section{Convergence time} \label{sec:Convergence}
\vspace{-3mm}\HRule

%    \input{sections/liveperf}

\clearpage
\section{Conclusions} \label{sec:Conclusions}
\vspace{-3mm}\HRule

%    \input{sections/conclusion}

\bibliographystyle{plain}

\begin{thebibliography}{99}

    \bibitem{bib:BATMAN} The \batman\ protocol:
    \url{http://www.open-mesh.org/}

    \bibitem{bib:OLSR} The \olsr\ protocol:
    \url{http://www.olsr.org/}

%    \bibitem{bib:WirelessSpec} IEEE Std 802.11\texttrademark 2007, {\it
%        Part 11: Wireless LAN Medium Access Control (MAC) and Physical
%        Layer (PHY) Specifications}
%
%    % TODO: add info about artelatex and other used documents.
%
%    \bibitem{} Eric W. Weisstein,  ``Confidence Interval,'' From
%    MathWorld--A Wolfram Web Resource,
%    \url{http://mathworld.wolfram.com/ConfidenceInterval.html}

\end{thebibliography}

\end{document}
