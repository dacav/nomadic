\clearpage
\subsection{Test introduction}
    Beyond throughput, another measure of interest when analyzing the
    performance of a network is the latency. We've seen in the
    previous section that the presence of the \emph{mesh} routing
    protocol do not seem to affect the overall throughput in the small
    environment at our disposal. An obvious question which could be
    raised is if the same is true also when taking into account latency.

    To answer this question we performed similar test to the one
    described above. For each topology and routing strategy we
    performed a latency test by \emph{pinging} the sink node from the
    destination. Each measurement lasted 600 seconds.

\subsection{Results}
The results we obtained are summarized in
Picture~\ref{pic:Latency}. As it's clear from the image, we have a
situation analogous to the one of throughput. The difference between
the three routing strategy is negligible.

Again an interesting note can be made on the behavior of \olsr. Even
if in the \emph{direct} topology it obtained the best result (though
with really spread min-max bounds) in the remaining experiments it
achieves worse performance. \batman\ instead is more constant and
always scores results comparable with the \emph{static} setting.

\Picture{images/latency_plot}
        {0.7 \columnwidth}
        {Impact of the meshing protocols on the transmission latency. The
         three triples of boxplots show the latency performances of,
         respectively, \emph{direct link}, \emph{1-hop} and \emph{2-hops}
         topology. Each triple shows the performance with \emph{static
         routes} (first boxplot), \emph{\batman} (second boxplot) and
         \emph{\olsr} (third boxplot).}
        {pic:Latency}
